\documentclass{includes/Thesis}

\addbibresource{parts/refs.bib}

\begin{thesis}[
    title = Анализ обработки исключений для языков Java и Kotlin в статическом анализаторе Svace,
    year = 2022,
    %
    authorGroup = БПИ182,
    authorName = В. О. Афанасьев,
    %
    academicTeacherTitle = {руководитель департамента <<Программная инженерия>>, доцент},
    academicTeacherName = С. А. Лебедев,
    %
    hasAcademicCoteacher,
    academicCoteacherTitle = преподаватель базовой кафедры\\<<Системное программирование>> ИСП РАН в НИУ ВШЭ,
    academicCoteacherName = А. Е. Волков,
    %
    isAcademic,
    UDC = 004.05,
    %
    hasConsultant,
    consultantTitle = младший научный сотрудник Института системного программирования РАН,
    consultantName = С. А. Поляков,
    %
    keywordsRu = статический анализ; поиск ошибок; обработка исключений; Java; Kotlin; JVM; байткод,
    keywordsEn = static analysis; search for defects; exception handling; Java; Kotlin; JVM; bytecode,
]

    \setAbstractResource{parts/abstract-ru}{parts/abstract-en}

    \setTerminologyResource{parts/terminology}

    \setIntroResource{parts/intro}

    \addChapter{Обзор источников}{parts/chapter1}
    \addChapter{Какая-нибудь ещё глава}{parts/chapter2}

    \setConclusionResource{parts/conclusion}

    \addAppendix{Пример приложения}{parts/appendix-example-1}
    \addAppendix{Ещё один пример приложения}{parts/appendix-example-2}

\end{thesis}
